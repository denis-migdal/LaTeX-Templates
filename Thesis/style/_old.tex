% *************** Document de style ***************

% *************** chargement des packages ***************

\usepackage{amsmath}
\usepackage{amsthm}
\usepackage{amsfonts}
\usepackage{amssymb}

\usepackage{graphicx}
\usepackage{lmodern} % Required for specifying colors by name
\usepackage{setspace}
\usepackage{textcomp}
\usepackage[numbers]{natbib}

\usepackage{tabularx}

%To make clear definition
\newtheorem{quickDef}{Definition}
\newtheorem{rmq}{Remark}

%This package permits to use vectorial font (acrobat reader compatibility)
\usepackage[cyr]{aeguill}



% paquets pour tableaux, figures...
\usepackage[dvipsnames,table,hyperref]{xcolor}
%\usepackage{longtable}
\usepackage{graphicx}
\usepackage{multirow}

% pour avoir des paragraphes justifi�s correctement
\usepackage{microtype} 

% pour faire des "cholies" figures
\usepackage{tikz}
\usetikzlibrary{arrows,decorations,backgrounds,shapes}

\usepackage{ifpdf}

% pour citer des adresses web
\usepackage{url}

% a changer suivant la version (d�finitive ou pour relecture ^^)
\def\baselinestretch{1.2} % simple interligne
%\def\baselinestretch{1.56}  % double interligne

% comme con nom l'indique
\fixpdflayout

% pour utiliser des lettrines (premiere lettre en grand ^^)
\usepackage{lettrine}
\newcommand{\malettrine}[2]{\lettrine[lines=2, lhang=0.33, loversize=0.25]{#1}{#2}} % format de la lettrine
\renewcommand{\LettrineFontHook}{\color [gray]{0.5}} % couleur de la lettrine

% *************** Permet d'ajouter la page des citations dans la biblio ***************
%usepackage{citeref}
%\renewcommand{\bibitempages}[1]{\newblock {\scriptsize [\mbox{cité p.\ }#1]}}


\usepackage{todonotes}


%pour corriger des incompatibilit�s entre 'm�moire' et 'hyperref'
\usepackage{memhfixc}


% *************** Autre ***************
% divers...

% un style de list personalis�...
\newenvironment{malist}
{ \begin{list}%
    {$\bullet$}%
    {\setlength{\labelwidth}{30pt}%
     \setlength{\leftmargin}{35pt}%
     \setlength{\itemsep}{\parsep}}}%
{ \end{list} }

% je ne sais plus pourquoi... pour corriger une erreur...
\newsubfloat{figure}

% \bf non g�r� par 'memoire', donc...
\renewcommand{\bf}{\textbf}

% pour avoir des guillements francais bien g�r�
\renewcommand{\og}{\og\xspace} % ouverture 
\renewcommand{\fg}{\fg\xspace} % fermeture

% pour faire des note (avec paquets todonotes), suffit de faire \note{sjhfjshfhfjh} pour que ca s'affiche sur fond vert clair
\newcommand{\note}[1]{\todo[inline,color=green!40]{#1}}

% raccourcie, pour citer Truc \etal 
\newcommand{\etal}{\emph{et coll.}\xspace} % et en francais, c'est et coll., pas et al.

% pour changer l'affiche des titres des figures
\makeatletter
\renewcommand{\fnum@figure}[1]{\figurename~\thefigure~-- \sffamily}
\makeatother

\usepackage[plainpages=false,pdfpagelabels,bookmarksnumbered, %
        colorlinks=true, % impression -> false // en ligne -> true % important pour le nombre de page couleur ! ca coute chere !
        linkcolor=Sepia, %
        citecolor=Sepia, %
        filecolor=Maroon, %
        urlcolor=Blue, %
        pdftex, %
        unicode, %
        backref=page]{hyperref} 

\renewcommand*{\backref}[1]{}
\renewcommand*{\backrefalt}[4]{%
    \ifcase #1 [Non cité.]%
    \or        [Cité~page~#2.]%
    \else      [Cité~pages~#2.]%
    \fi}


\renewcommand*{\backreftwosep}{ et~}
\renewcommand*{\backreflastsep}{, et~}

\usepackage{amsthm}
\newtheorem{definition}{Definition}
\usepackage{tabularx}
\usepackage{amssymb}

\usepackage{pifont}% http://ctan.org/pkg/pifont
\newcommand{\cmark}{\ding{51}}%
\newcommand{\xmark}{\ding{55}}%

\newcommand{\mcrot}[4]{\multicolumn{#1}{#2}{\rlap{\rotatebox{#3}{#4}~}}}

%\usepackage{algorithm}
\usepackage{algpseudocode}
\usepackage{pdflscape}

\usepackage[Algorithme]{algorithm}

\renewcommand{\algorithmicrequire}{\textbf{Require:}}
\renewcommand{\algorithmicensure}{\textbf{Ensure:}}
\renewcommand{\algorithmiccomment}[1]{\{#1\}}
\renewcommand{\algorithmicend}{\textbf{fin}}
\renewcommand{\algorithmicif}{\textbf{si}}
\renewcommand{\algorithmicthen}{\textbf{alors}}
\renewcommand{\algorithmicelse}{\textbf{sinon}}
\renewcommand{\algorithmicfor}{\textbf{pour}}
\renewcommand{\algorithmicforall}{\textbf{pour tous}}
\renewcommand{\algorithmicdo}{\textbf{faire}}
\renewcommand{\algorithmicwhile}{\textbf{while}}

% *************** Fin du style ***************


%\usepackage{gensymb}
%\usepackage{pdfpages}
%Author date and title
%\author{Julien Hatin}
%\date{10-12-2014}

%Bibliography style
%\bibliographystyle{plainnat-fr}
%\usepackage{bibentry}

%\usepackage{afterpage}

%\newcommand\blankpage{%
%    \null
%    \thispagestyle{empty}%
%    \addtocounter{page}{-1}%
%    \newpage}

%%%%%%%%%%%%%%%%%%%%%%%%%%%%%%%%%%%%%

%\usepackage[utf8]{inputenc}
%\usepackage{geometry}
%\usepackage{hyperref}
%\usepackage{multirow}
%\usepackage{calrsfs} % fancy \mathcal
%\geometry{hmargin=1.5cm,vmargin=1.5cm}%v=1.5 - h=2.5

%\input{src/commands/thm.tex}
%\input{src/commands/notes.tex}
%\input{src/commands/compresslist.tex}
%\input{src/commands/notableentry.tex}