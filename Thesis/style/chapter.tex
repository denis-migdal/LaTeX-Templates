% voir la ref du 'memoire' pour plus d'info
% permet de changer le style des titre de chapitre, section, paragraphes...

\chapterstyle{bianchi}

\setsecheadstyle{\Large\sffamily\bfseries}
\setsubsecheadstyle{\large\sffamily\bfseries}
\setsubsubsecheadstyle{\normalfont\sffamily\bfseries}
\setparaheadstyle{\normalfont\sffamily}


\makeevenhead{headings}{\thepage}{}{\small\slshape\leftmark}
\makeoddhead{headings}{\small\slshape\rightmark}{}{\thepage}


\setafterparaskip{1ex plus .5ex}

%%%%%%%%%%% MINI TABLE OF CONTENT %%%%%%%%%%%
% Rq Replace minitoc (incompatible with memoir)

\usepackage{titletoc}
\usepackage{titlesec}

\newcommand\partialtocname{Contents}
\newcommand\ToCrule{\noindent\rule[5pt]{\textwidth}{1.3pt}}
\newcommand\ToCtitle{{\large\bfseries\partialtocname}\vskip2pt\ToCrule}
\makeatletter
\newcommand\Mprintcontents{%
  \ToCtitle
  \ttl@printlist[chapters]{toc}{}{1}{}\par\nobreak
  \ToCrule}
\makeatother

\newcommand{\chaptertoc}{
    \startcontents[chapters]
    \Mprintcontents
}
    
%%%%%%%%%%% CHAPTER ABSTRACT %%%%%%%%%%%
% gestion de la commande epigraph
\setlength{\epigraphwidth}{0.87\textwidth}
\setlength{\epigraphrule}{0pt}
\setlength{\beforeepigraphskip}{1\baselineskip}
\setlength{\afterepigraphskip}{0\baselineskip}

\newcommand{\epitext}{\raggedleft\sffamily\itshape}
\newcommand{\epiauthor}{\sffamily\scshape ---~}
\newcommand{\epititle}{\sffamily\itshape}
\newcommand{\epidate}{\sffamily\scshape}
\newcommand{\episkip}{\medskip}

\newcommand{\myepigraph}[4]{%
	\epigraph{\epitext \raggedleft #1\episkip}{\epiauthor #2\\\epititle #3 \epidate(#4)}\noindent}


\NewEnviron{chapterabstract}{
    \epigraph{\parbox{\linewidth}{\emph{\BODY}}}{}{}
}

\newcommand{\chapterstar}[1]{
    \chapter*{#1}
    \addcontentsline{toc}{chapter}{#1}
}